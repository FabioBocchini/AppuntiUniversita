\section{Come Riconoscere un Modello di Coda}
Un modello di coda secondo la notazione di Kendall è così rappresentato:
\begin{center}
    $A/b/c/n/p/z$
\end{center}

dove:

\begin{itemize}
    \item $A$: indica la distribuzione del tempo di inter-arrivo
    \item $b$: indica la distribuzione del tempo di servizio $T_s$
    \item $c$: indica il numero di serventi
    \item $n$: indica la dimensione della coda
    \item $p$: indica la dimensione della popolazione
    \item $Z$: indica la disciplina di servizio
\end{itemize}
Tale notazione si semplifica in $A/b/c$ nel caso in cui la dimensione della
popolazione e della coda sono infinite e la disciplina di servizio segue la
logica FIFO ($n = p = \infty$ e $Z = \text{FIFO}$).\\
Per quanto riguarda i possibili valori di $A$, $b$ e $c$:

\begin{itemize}
    \item $A$ e $b$: può assumere i valori D (distribuzione deterministica o
          costante), M (distribuzione esponenziale negativa), G (distribuzione
          generale), $H_h$ (distribuzione iperesponenziale), $E_k$ (l'Erlangiana
          a k stadi)
    \item $c$: 1 o m, dove 1 indica un singolo servente e m indica che ci sono
          serventi multipli. Non importa inizialmente specificare quanto è m, ma
          per le formule successive il valore va sostituito con il numero esatto
          di serventi.
\end{itemize}

\subsection{Consigli}
Solitamente negli esercizi è sottinteso che la dimensione della popolazione e
della coda sono infinite (non lo sono soltanto nel caso in cui viene specificato
diversamente), lo stesso vale per la gestione del servizio che è sempre FIFO
(salvo casi estremi che devono essere specificati).\\
Per quanto riguarda le distribuzioni degli interarrivi e del servizio: essi sono
sempre specificati e nei soli casi in cui non viene esplicitato il tipo di
distribuzione (che ovviamente può essere diverso per interarrivo e servizio) si
considera la distribuzione generale G.\\
Nei pochi casi in cui la distribuzione è deterministica è sempre specificato, ad
esempio viene detto che il tempo è costante.

\section{Parametri Fondamentali}

\begin{itemize}
    \item $\Delta$: Tempo di Inter-arrivo (il tempo che intercorre tra un arrivo
          e il successivo)
    \item $w$: Numero di utenti in coda
    \item $t_w$: Tempo di Attesa in Coda
    \item $s$: Numero di Utenti in Servizio
    \item $t_s$: Tempo di Servizio
    \item $q$: Numero di Utenti nel Sistema
    \item $t_q$: Tempo di Risposta
\end{itemize}

\paragraph{N.B.}
\begin{itemize}
    \item Tutti i \textbf{Tempi} vanno espressi in \textbf{minuti},
    \item tutti i valori precedenti sono \textbf{interi} e \textbf{maggiori o
              uguali} a 0,
    \item $0 \leq s \leq c$.
    \item \textbf{stare bene a tenti a se la chiede in ore o minuti \normalfont{\emoji{pig}}}
\end{itemize}

\section{Come Calcolare i Parametri Base} \label{parametri-base}

\begin{itemize}
    \item \textit{Tempo Medio di Servizio} $T_s = \frac{1}{\mu}$
    \item \textit{Tempo medio di Inter-arrivi} $\mu = \frac{1}{T_s}$
    \item \textit{Tasso medio di Arrivi} $\lambda = \Delta^{-1}$
    \item \textit{Intensità del Traffico} $\rho = \frac{\lambda}{\mu}$
\end{itemize}

\subsection{Domande} %%TODO: ricontrollare, n forse sono gli utenti

\begin{itemize}
    \item \textit{Qual è la distribuzione di probabilità del numero di arrivi?}
          \vspace{0.3cm}
          \\Nel caso in cui abbiamo i \textbf{tempi di inter-arrivo
              Esponenziali} allora avremo i \textbf{tempi di arrivo} con
          distribuzione di \textbf{Poisson}:
          \begin{itemize}
              \item La \textbf{densità di probabilità} del numero di
                    \textbf{arrivi} si calcola con:
                    $$P_d = \frac{e^{-\lambda} \lambda^n}{n!}$$
              \item Se si ha un \textbf{blocco del sistema} per un \textbf{tempo
                        $t$}, la \textbf{probabilità che ci siano $n$ utenti} è:
                    $$P_d = \frac{e^{-t\lambda} \left ( t\lambda \right
                            )^n}{n!}$$
          \end{itemize}

    \item \textit{Qual è la distribuzione di probabilità dei tempi di
              Inter-Arrivo?}
          \vspace{0.3cm}
          \\Nel caso in cui abbiamo i \textbf{tempi di arrivo} con distribuzione
          di \textbf{Poisson} allora i \textbf{tempi di inter-arrivo} saranno
          \textbf{esponenziali}:
          \begin{itemize}
              \item La \textbf{densità di probabilità} dei tempi di inter-arrivo
                    si calcola con $$f(n) = \lambda e^{-\lambda n}$$
              \item Se si ha un \textbf{blocco del sistema} per un \textbf{tempo
                        $t$}, la \textbf{probabilità che ci siano $n$ utenti} è:
                    $$f(n) = t\lambda e^{-t\lambda n}$$
              \item La \textbf{funzione di distribuzione} è: $$F(n) = 1 -
                        e^{-\lambda n}$$
          \end{itemize}
\end{itemize}

\section{Condizione di Stazionarietà}

$$
    \rho \le 1
$$

Controllare bene (\textit{STARE BENE A TENTI}) che la condizione di
Stazionarietà sia verificata altrimenti l'esercizio non si può continuare.
