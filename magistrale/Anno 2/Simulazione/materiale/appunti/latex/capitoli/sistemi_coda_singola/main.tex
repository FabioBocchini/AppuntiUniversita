\chapter{Sistemi a Coda Singola}
%TODO: rimuovere i * dalle operazioni matematiche
\section{Come Riconoscere un Modello di Coda}
Un modello di coda secondo la notazione di Kendall è così rappresentato:
\begin{center}
    $A/b/c/n/p/z$
\end{center}

dove:

\begin{itemize}
    \item $A$: indica la distribuzione del tempo di inter-arrivo
    \item $b$: indica la distribuzione del tempo di servizio $T_s$
    \item $c$: indica il numero di serventi
    \item $n$: indica la dimensione della coda
    \item $p$: indica la dimensione della popolazione
    \item $Z$: indica la disciplina di servizio
\end{itemize}
Tale notazione si semplifica in $A/b/c$ nel caso in cui la dimensione della
popolazione e della coda sono infinite e la disciplina di servizio segue la
logica FIFO ($n = p = \infty$ e $Z = \text{FIFO}$).\\
Per quanto riguarda i possibili valori di $A$, $b$ e $c$:

\begin{itemize}
    \item $A$ e $b$: può assumere i valori D (distribuzione deterministica o
          costante), M (distribuzione esponenziale negativa), G (distribuzione
          generale), $H_h$ (distribuzione iperesponenziale), $E_k$ (l'Erlangiana
          a k stadi)
    \item $c$: 1 o m, dove 1 indica un singolo servente e m indica che ci sono
          serventi multipli. Non importa inizialmente specificare quanto è m, ma
          per le formule successive il valore va sostituito con il numero esatto
          di serventi.
\end{itemize}

\subsection*{Consigli}
Solitamente negli esercizi è sottinteso che la dimensione della popolazione e
della coda sono infinite (non lo sono soltanto nel caso in cui viene specificato
diversamente), lo stesso vale per la gestione del servizio che è sempre FIFO
(salvo casi estremi che devono essere specificati).\\
Per quanto riguarda le distribuzioni degli interarrivi e del servizio: essi sono
sempre specificati e nei soli casi in cui non viene esplicitato il tipo di
distribuzione (che ovviamente può essere diverso per interarrivo e servizio) si
considera la distribuzione generale G.\\
Nei pochi casi in cui la distribuzione è deterministica è sempre specificato, ad
esempio viene detto che il tempo è costante.

\section{Parametri Fondamentali}

\begin{itemize}
    \item $\Delta$: Tempo di Inter-arrivo (il tempo che intercorre tra un arrivo
          e il successivo)
    \item $w$: Numero di utenti in coda
    \item $t_w$: Tempo di Attesa in Coda
    \item $s$: Numero di Utenti in Servizio
    \item $t_s$: Tempo di Servizio
    \item $q$: Numero di Utenti nel Sistema
    \item $t_q$: Tempo di Risposta
\end{itemize}

\paragraph{N.B.}
\begin{itemize}
    \item Tutti i \textbf{Tempi} vanno espressi in \textbf{minuti},
    \item tutti i valori precedenti sono \textbf{interi} e \textbf{maggiori o
              uguali} a 0,
    \item $0 \leq s \leq c$.
    \item \textbf{stare bene a tenti a se la chiede in ore o minuti :pig:}
\end{itemize}

\section{Come Calcolare i Parametri Base} \label{parametri-base}

\begin{itemize}
    \item \textit{Tempo Medio di Servizio} $T_s = \frac{1}{\mu}$
    \item \textit{Tempo medio di Inter-arrivi} $\mu = \frac{1}{T_s}$
    \item \textit{Tasso medio di Arrivi} $\lambda = \Delta^{-1}$
    \item \textit{Intensità del Traffico} $\rho = \frac{\lambda}{\mu}$
\end{itemize}

\section{Condizione di Stazionarietà}

$$
    \rho \le 1
$$

Controllare bene (\textit{STARE BENE A TENTI}) che la condizione di
Stazionarietà sia verificata altrimenti l'esercizio non si può continuare.

\section{Formule}


\section{Capire come gestire la variazione della distribuzione e del suo modello di coda}

\section{M/M/1}

Sistema aperto denotato da un singolo servente:

\begin{itemize}
    \item Distribuzione del Tempo di Inter-arrivo Esponenziale con parametro
          $\lambda$
    \item Tempo di Servizio Esponenziale di parametro $\mu$
\end{itemize}

\subsection{Parametri}

\begin{itemize}
    \item Numero di Utenti Medio: $N = \frac{\rho}{1 - \rho} = \lambda R$
    \item Numero Medio di Utenti in Coda: $W = N - \rho = \frac{\rho^2}{1-\rho}$
    \item Tempo Medio di Risposta: $R = \frac{\frac{1}{\mu}}{1 - \rho} = T_s +
              T_w = \frac{N}{\lambda}$
    \item Tempo di Attesa Medio in Coda: $T_w = \frac{\frac{\rho}{\mu}}{1 -
                  \rho} = R - T_s$
    \item Probabilità di Osservare almeno $k$ utenti in un Sistema in condizione
          di Stazionarietà: $ = \rho^k$
    \item Probabilità di avere $0$ utenti nel sistema: $\pi_0 = 1 - \rho$
    \item Probabilità di avere $k$ utenti nel sistema: $\pi_k = \rho^k \pi_0 =
              \rho^k (1 - \rho)$
\end{itemize}

\section{M/M/m}

Sistema aperto dotato di $m$ serventi:

\begin{itemize}
    \item Distribuzione del Tempo di Arrivo Poissoniano con parametro $\lambda$
    \item Distribuzione del Tempo di Servizio Esponenziale con parametro $\mu$
\end{itemize}

\subsection{Parametri}

\begin{itemize}
    \item Numero di Servienti: $m$
    \item Tempo Medio di Servizio: $T_s$ (vedi \ref{parametri-base})
    \item Tasso Medio di Arrivi $\lambda$ (vedi \ref{parametri-base})
    \item Tempo Medio di Inter-Arrivo: $\mu$ (vedi \ref{parametri-base})
    \item Intensità del Traffico: $\rho = \frac{\lambda}{m \mu}$
    \item Probabilità di avere $0$ utenti nel sistema:
          $$\pi_0 = \left [ \sum_{k=0}^{m-1} \left ( \frac{(m \rho)^k}{k!}
                  \right ) + \frac{(m \rho)^m}{m!} \frac{1}{1-\rho} \right ]^{-1}$$
    \item Probabilità di avere $k$ utenti nel sistema:
          \begin{itemize}
              \item[\emoji{orangutan}] se $1 \leq k \leq m$ $$\pi_k = \frac{(m
                          \rho)^k}{k!} \pi_0$$
              \item[\emoji{gorilla}] se $k > m$ $$\pi_k = \frac{m^m \rho^k}{m!}
                      \pi_0$$
          \end{itemize}
    \item Numero Medio di Serventi Occupati: $$E[s] = \sum_{k=0}^{m-1} \left (
              k\pi_k \right ) + \frac{m\pi_m}{1-\rho}  = m \rho = \frac{\lambda}{\mu}$$
    \item Numero di Utenti Medio: $N = m \rho + \pi_m \frac{\rho}{(1-\rho)^2}$
    \item Numero di Utenti Medio in Coda: $W = \pi_m \frac{\rho}{(1-\rho)^2}$
    \item Tempo Medio di Risposta: $R = \frac{N}{\lambda} = \frac{m \rho +
                  W}{\lambda}$
    \item Tempo di Attesa in Coda: $$T_w = \frac{\pi_m}{m\mu (1-\rho)^2}$$
    \item Tempo di Utilizzo (tempo in cui si sta bene a tenti): $U = 1 - \pi_0 =
              \rho$
    \item Tempo di Non Utilizzo: $\hat{U} = 1 - U$
    \item Probabilità che un Utente in Arrivo trovi tutti i serventi occupati:
          $$Prob_{coda} = \sum_{k=m}^{+\infty} \pi_k = \pi_0 \frac{(m\rho)^m}{m!}
              \frac{1}{1-\rho}$$
    \item Probabilità che un Utente in Arrivo Non trovi una coda:
          $$\hat{Prob_{coda}} = 1 - Prob_{coda}$$
\end{itemize}

\section{M/M/\texorpdfstring{$\infty$}{infinito}}

Sistema aperto con infiniti servienti:

\begin{itemize}
    \item Distribuzione del Tempo di Arrivo Poissoniano di parametro $\lambda$
    \item Distribuzione del Tempo di Servizio Esponenziale di parametro $\mu$
\end{itemize}

\subsection{Parametri}

\begin{itemize}
    \item Intensità del Traffico: $\rho$ (vedi \ref{parametri-base})
    \item Probabilità di avere $k$ utenti, che coincide (in questo caso
          specifico) con la Probabilità di avere $k$ serventi occupati: $$\pi_k =
              \frac{\rho^k}{k!} e^{-\rho}$$ con $k \geq 0$
    \item Numero Medio di Utenti: $N = \rho$
    \item Tempo Medio di Risposta, che coincide con il Tempo Medio di Servizio:
          $$R = T_s = \frac{1}{\mu}$$
\end{itemize}

\section{M/M/1/K (dimensione coda finita)}

\section{M/M/1//M (dimensione popolazione finita)}

\section{M/G/1} \label{mg1}

Sistema aperto con un singolo servente:

\begin{itemize}
    \item Distribuzione del Tempo di Inter-Arrivo Esponenziale con parametro
          $\lambda$
    \item Distribuzione del Tempo di Servizio degli Utenti Indipendente con
          Distribuzione Generale
\end{itemize}

\subsection{Parametri}

\begin{itemize}
    \item Per quelli di base vedere \ref{parametri-base}
    \item Numero Medio di Utenti (formula di \textit{Khintchine-Pollaczk} \emoji{man-with-veil-dark-skin-tone}): $$N
              = \rho + \frac{\rho^2 (1 + C^2_B)}{2 (1-\rho)}$$ dove :
          \begin{itemize}
              \item $C_B = \sigma \mu$ (Coefficiente di Variazione)
              \item $\sigma = \sqrt{\text{Varianza}}$ (Deviazione Standard)
          \end{itemize}
    \item Tempo Medio di Risposta di un lavoro: $R = \frac{N}{\lambda}$
    \item Tempo Medio di Attesa in Coda: $W = \lambda T_w = N - \rho$
    \item Tempo di Attesa in Coda: $T_w = \frac{N -\rho}{\lambda}$
\end{itemize}

\paragraph{N.B.}
\begin{itemize}
    \item Se $\rho = 1$ e quindi il sistema è \textbf{congestionato}, allora gli
          indici medi $N, W, R, T_w$ tendono a crescere senza limite.
\end{itemize}

\section{M/D/1}

Versione di $M/G/1$ con Distribuzione del Tempo di Servizio \textit{Deterministico}:

\begin{itemize}
    \item Distribuzione del Tempo di Inter-Arrivo Esponenziale con parametro
          $\lambda$
    \item Distribuzione del Tempo di Servizio degli Utenti Indipendente con
          Distribuzione Deterministica
\end{itemize}
\subsection{Parametri}

\begin{itemize}
    \item Valore Medio degli Utenti nel Sistema: $$N = \rho + \frac{\rho^2}{2 (1-\rho)}$$
    \item Numero di Utenti Medio in Attesa: $$W = \frac{\rho^2}{2(1-\rho)}$$
\end{itemize}

\paragraph{N.B.}
\begin{itemize}
    \item Tutti i parametri che non sono stati elencati sono calcolati come scritto in \ref{mg1}
    \item Se $\rho = 1$ e quindi il sistema è \textbf{congestionato}, allora gli
          indici medi $N, W, R, T_w$ tendono a crescere senza limite.
\end{itemize}